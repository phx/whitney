\section*{Appendix B: Gauge Symmetry Integration}
\label{app:gauge_symmetry}

\subsection*{B.1 Emergence of Standard Model Gauge Groups}
\label{subsec:sm_gauge_groups}

We demonstrate how the SM gauge group $SU(3)\times SU(2)\times U(1)$ emerges naturally at low energies from our fractal structure. Consider the gauge transformation:

\[
\mathcal{G}(x, E) = \exp\left(i\sum_{a} \omega^a(x) T^a(E)\right)
\]

where $T^a(E)$ are the energy-dependent generators satisfying:

\[
[T^a(E), T^b(E)] = if^{abc}(E)T^c(E)
\]

The structure constants $f^{abc}(E)$ evolve with energy scale through our fractal recursion:

\[
f^{abc}(E) = f^{abc}_0 + \sum_{n=1}^{\infty} \alpha^n h(n,E) f^{abc}_n
\]

At $E \sim M_Z$, the series truncates to reproduce the familiar SM structure constants.

\subsection*{B.2 Preservation of Gauge Invariance}
\label{subsec:gauge_invariance}

To prove gauge invariance is preserved across all scales, we show that under a gauge transformation:

\[
\Psi_n \rightarrow \mathcal{G}(x,E)\Psi_n
\]

The full field equation remains invariant:

\[
\mathcal{F}(x,t,E) \rightarrow \mathcal{F}(x,t,E)
\]

This follows from the fractal structure of our coupling constants...